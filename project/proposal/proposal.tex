\documentclass{article}
\usepackage{amsmath}
\usepackage{amsfonts}
\usepackage{tikz}
\usepackage{algorithm}
\usepackage{algpseudocode}

\textwidth=7.6in
\textheight=9.9in
\topmargin=-.9in
\headheight=0in
\headsep=.5in
\hoffset=-1.5in
\setlength\parindent{0pt}

\begin{document}
\begin{center}
    \textbf{Final Project Proposal} \\[0.25ex]
    Calvin Walker
\end{center}
% \begin{itemize}
%     \item What problem are you interested in addressing?
%     \item How does this problem relate to the course material?
%     \item What approach(es) will you take to address this problem?
%     \item Why are these approaches appropriate?
%     \item What technical challenges do you expect to face?
%     \item What metric will you use to evaluate success?
%     \item What, if any, dataset(s) will you use?
%     \item What, if any, software will you use?
%     \item What is your anticipated timeline?
% \end{itemize}
\textbf{Problem}: Classic methods for causal inference in economic and other social science research often rely on the Stable Unit Treatment Value Assumption (SUTVA), which requires that the response of an individual only depends on the treatment to which they were assigned, i.e. if a unit's friend is in the treatment group, their treatment only effects them. 
However, there are numerous situations of interest to researchers where this may not be a plausible assumption, and being able to separate the effects of the treatment assigment and peer influence have important implications. \\[0.5ex]
A classic method to deal with situations where SUTVA may not hold, is to perform clustering, so that the assumption only needs to be true between classes instead of students, or schools instead of classes, for instance, if the treatment were extra test prep. While this is certainly a more plausible assumption, now we are limited to learning about the intervention's effect across schools, and must have two seperate populations that are good counterfactuals for eachother. \\[0.5ex]
Graphs are a common, and intuitive way to express social interactions between individuals—offering an attractive model for learning, and properly acounting for peer influence in experimental settings. There is a growing body of literature devoted to the analysis of social networks expressed as graphs, and uses in experimental settings. The questions motivating my research are: How can we use probablistic graphical models to represent influence between individuals in experimental settings? How might we learn the potentials for influence between individuals? And how can we use this graphical model to perform inference in experimental settings that allows for seperation of treatment effects, and the effect of peer influence.\\[1.0ex]  
\textbf{Approach}: My approach to the problem will be to model peer influence using a probabilistic graphical model such as a markov random field, where edges and potentials represent social connection and influence between individuals in the graph. I will investigate ways to learn the potentials in this graph, and how to use this learned graph to estimate the effect of peer influence on the outcome of interest. Formulating this explicitly will most likely be the greatest technical challenge, but it seems achievable using the course material, and building off of the existing literature.  \\[1.0ex]
\textbf{Evaluating Success}: I will first evaluate the performance of my approach by testing it on a semi-synthetic dataset, which is common in the related literature. In practice, this will mean taking an existing dataset of a social network/graph, and simulating treatment effects and peer influence using some data generating process, so see if my approach can recover the true effects of treatment and peer influence. Next, I hope to test my approach empirically on experimental data, but I am still searching for a suitable dataset. The only software I expect to use is Python for data manipulation, running my experiments, etc. \\[1.0ex] 
\textbf{Timeline}: Given that there is about a month remaining before the final project presentations, I hope to spend the next week continuing my literature review, choosing both datasets, and finalizing the general structure of my approach to the problem. Then, the next couple weeks will be spent explicitly defining my solution, and running the necessary experiments on the data to test its efficacy. Finally, the last week will be spent fixing errors, and summarizing my findings.

\end{document}